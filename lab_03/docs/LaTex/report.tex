% !TeX document-id = {c68f4be8-c497-43e0-82df-e9ebfbea9577}
% !TeX TXS-program:pdflatex = pdflatex -synctex=1 -interaction=nonstopmode --shell-escape %.tex
% новая команда \RNumb для вывода римских цифр
\documentclass[a4paper,12pt]{article}
\usepackage{amssymb}
\usepackage{amsmath}
\usepackage{amsthm} 
\usepackage{caption}
\usepackage{misccorr}
\usepackage[noadjust]{cite}
\usepackage{cmap} 
\usepackage[utf8]{inputenc}
\usepackage[T2A]{fontenc}
\usepackage[english, russian]{babel}
\usepackage{graphics}
\usepackage{graphicx}
\usepackage{textcomp}
\usepackage{verbatim}
\usepackage{makeidx}
\usepackage{geometry}
\usepackage{float}
\usepackage{bm}
\usepackage{esint}
\usepackage{mathtools}
\usepackage{graphicx}
\usepackage{listings}
\usepackage{courier}

\lstset{basicstyle=\fontsize{10}{10}\selectfont,breaklines=true}

\newcommand{\specchapter}[1]{\chapter*{#1}\addcontentsline{toc}{chapter}{#1}}
\newcommand{\specsection}[1]{\section*{#1}\addcontentsline{toc}{section}{#1}}
\newcommand{\specsubsection}[1]{\subsection*{#1}\addcontentsline{toc}{subsection}{#1}}
\newcommand{\RNumb}[1]{\uppercase\expandafter{\romannumeral #1\relax}}
\newcommand{\jj}{\righthyphenmin=20 \justifying}


% геометрия
\geometry{pdftex, left = 2cm, right = 2cm, top = 2.5cm, bottom = 2.5cm}

\setcounter{tocdepth}{4} % фикс переноса 
\righthyphenmin = 2
\tolerance = 2048

\begin{document}
\thispagestyle{empty}

\noindent \begin{minipage}{0.15\textwidth}
	\includegraphics[width=\linewidth]{b_logo}
\end{minipage}
\noindent\begin{minipage}{0.9\textwidth}\centering
	\textbf{Министерство науки и высшего образования Российской Федерации}\\
	\textbf{Федеральное государственное бюджетное образовательное учреждение высшего образования}\\
	\textbf{«Московский государственный технический университет имени Н.Э.~Баумана}\\
	\textbf{(национальный исследовательский университет)»}\\
	\textbf{(МГТУ им. Н.Э.~Баумана)}
\end{minipage}

\noindent\rule{18cm}{3pt}
\newline\newline
\noindent ФАКУЛЬТЕТ $\underline{\text{«Информатика и системы управления»}}$ \newline\newline
\noindent КАФЕДРА $\underline{\text{«Программное обеспечение ЭВМ и информационные технологии»}}$\newline\newline\newline\newline\newline\newline\newline


\begin{center}
	\noindent\begin{minipage}{1.3\textwidth}\centering
	\Large\textbf{  Лабораторная работа № 3}\newline
	\textbf{по дисциплине "Вычислительные алгоритмы"}\newline\newline\newline
	\end{minipage}
\end{center}

\noindent\textbf{Тема} $\underline{\text{Интерполяция сплайнами.}}$\newline\newline
\noindent\textbf{Студент} $\underline{\text{Романов А.В.}}$\newline\newline
\noindent\textbf{Группа} $\underline{\text{ИУ7-43Б}}$\newline\newline
\noindent\textbf{Оценка (баллы)} $\underline{\text{~~~~~~~~~~~~~~~~~~~~~~~~~~~}}$\newline\newline
\noindent\textbf{Преподаватель} $\underline{\text{Градов В.М.}}$\newline
	
\begin{center}
	\vfill
	Москва~---~\the\year
~г.
\end{center}
\clearpage
\section{Задание}

\textbf{}Задана таблица значений функции вида $x$, $f(x)$. Провести интерполяцию сплайном на данной таблице, и найти значение $f(x)$. \newline

\noindent\textbf{Входные данные}: Таблица значений функции, значение точки по координате $X$. \newline
\noindent\textbf{Выходные данные}: значение функции $f(x)$. \newline

\section{Описание алгоритма}
\begin{flushleft}

Кубический сплайн — это кривая, состоящая из „состыкованных“ полиномов третьей степени
($y^{\RNumb{4}}$ ($x$) = 0). В точках стыковки значения и производные двух соседних полиномов равны. \newline \newline
\textbf{}Имеем:

\begin{center}
\begin{minipage}{0.6\textwidth}
$\varphi(x) = a_{i} + b_{i}(x - x_{i}) + c_{i}(x - x_{i})^2 + d_{i}(x - x_{i})^3$ \newline
$\varphi(x_{i-1}) = a_{i}$ \newline
$\varphi(x_{i}) = a_{i} + b_{i}h_{i} + c_{i}{h_{i}}^2 + d_{i}{h_{i}}^3$ \newline
$\varphi'(x_{i}) = и_{i} + 2с_{i}(x - x_{i - 1}) + 3d_{i}(x - x_{i - 1})^2 $ \newline
$\varphi''(x_{i}) = 2с_{i} + 6d_{i}(x_{i} - x_{i - 1})$ \newline
$c_{i} + 3d_{i}h_{i} = c_{i + 1}$
\end{minipage}
\end{center}

Получим СЛАУ с трехдиаганальной матрицей:

\begin{equation*}
\begin{cases}
c_{1} = 0\\
h_{i - 1}c_{i - 1} + 2(h_{i - 1} + h_{i})c_{i} + h_{i}c_{i + 1} = 3(\frac{{y_{i} - y_{i - 1}}}{h_{i}} - \frac{{y_{i - 1} - y_{i - 2}}}{h_{i}}) \\
c_{N + 1} = 0

\end{cases}
\end{equation*}
\newline \newline
\textbf{}Решается СЛАУ методом прогонки:

\textbf{1. } Находятся все прогочные коэффициенты по формулам:
\begin{center}
\begin{minipage}{0.6\textwidth}
$\xi_{i + 1} = \frac{D_{i}}{B_{i} - A_{i}\xi_{i}}$ \newline \newline
$\eta_{i + 1} = \frac{F_{i} + A_{i}\eta_{i}}{B_{i} - A_{i}\xi_{i}} $
\end{minipage}
\end{center}
\textbf{2. } При известном $y_{N}$ определяются все $y_{i}$ -- обратный ход. \newline
Приминительно к задаче поиска коэффициентов сплайна имеем $c_{i]} \Leftrightarrow y_{i}$ \newline

Обратный ход:
\begin{center}
\begin{minipage}{0.6\textwidth}
$c_{i} = \xi_{i + 1}c_{i + 1} + \eta_{i + 1}$, при $c_{N + 1} = 0$ и $c_{N} = \xi_{i + 1}$
\end{minipage}
\end{center}

При найденых c:

\begin{center}
\begin{minipage}{0.6\textwidth}
$a_{i} = y_{i - 1} $ \newline

$d_{i} = \frac{c_{i + 1} - c_{i}}{3h_{i}}d_{N} = -\frac{C_{N}}{3h_{N}}$ \newline

$b_{i} = \frac{y_{i} - y_{i - 1}}{h_{i}} - \frac{1}{3}h_{i}(c_{i + 1} + 2c_{i})$
\end{minipage}
\end{center}
\end{flushleft}


\clearpage % разрыв страницы

\section{Код программы}
\noindent\textbf{Файл Main.hs:}
\lstinputlisting[language=Haskell]{../../src/Main.hs}

\noindent\textbf{\newlineФайл Spline.hs:}
\lstinputlisting[language=Haskell]{../../src/Spline.hs}

\end{document}

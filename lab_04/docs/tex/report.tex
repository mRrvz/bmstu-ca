% !TeX document-id = {c68f4be8-c497-43e0-82df-e9ebfbea9577}
% !TeX TXS-program:pdflatex = pdflatex -synctex=1 -interaction=nonstopmode --shell-escape %.tex
% новая команда \RNumb для вывода римских цифр
\documentclass[a4paper,12pt]{article}
\usepackage{amssymb}
\usepackage{amsmath}
\usepackage{amsthm} 
\usepackage{caption}
\usepackage{misccorr}
\usepackage[noadjust]{cite}
\usepackage{cmap} 
\usepackage[utf8]{inputenc}
\usepackage[T2A]{fontenc}
\usepackage[english, russian]{babel}
\usepackage{graphics}
\usepackage{graphicx}
\usepackage{textcomp}
\usepackage{verbatim}
\usepackage{makeidx}
\usepackage{geometry}
\usepackage{float}
\usepackage{bm}
\usepackage{esint}
\usepackage{mathtools}
\usepackage{graphicx}
\usepackage{listings}
\usepackage{courier}
\usepackage{multirow}
\usepackage{graphicx}
\usepackage{pstricks}

\lstset{basicstyle=\fontsize{10}{10}\selectfont,breaklines=true}

\newcommand{\specchapter}[1]{\chapter*{#1}\addcontentsline{toc}{chapter}{#1}}
\newcommand{\specsection}[1]{\section*{#1}\addcontentsline{toc}{section}{#1}}
\newcommand{\specsubsection}[1]{\subsection*{#1}\addcontentsline{toc}{subsection}{#1}}
\newcommand{\RNumb}[1]{\uppercase\expandafter{\romannumeral #1\relax}}
\newcommand{\jj}{\righthyphenmin=20 \justifying}


% геометрия
\geometry{pdftex, left = 2cm, right = 2cm, top = 2.5cm, bottom = 2.5cm}

\setcounter{tocdepth}{4} % фикс переноса 
\righthyphenmin = 2
\tolerance = 2048

\begin{document}
\thispagestyle{empty}

\noindent \begin{minipage}{0.15\textwidth}
	\includegraphics[width=\linewidth]{b_logo}
\end{minipage}
\noindent\begin{minipage}{0.9\textwidth}\centering
	\textbf{Министерство науки и высшего образования Российской Федерации}\\
	\textbf{Федеральное государственное бюджетное образовательное учреждение высшего образования}\\
	\textbf{«Московский государственный технический университет имени Н.Э.~Баумана}\\
	\textbf{(национальный исследовательский университет)»}\\
	\textbf{(МГТУ им. Н.Э.~Баумана)}
\end{minipage}

\noindent\rule{18cm}{3pt}
\newline\newline
\noindent ФАКУЛЬТЕТ $\underline{\text{«Информатика и системы управления»}}$ \newline\newline
\noindent КАФЕДРА $\underline{\text{«Программное обеспечение ЭВМ и информационные технологии»}}$\newline\newline\newline\newline\newline\newline\newline


\begin{center}
	\noindent\begin{minipage}{1.3\textwidth}\centering
	\Large\textbf{  Лабораторная работа № 4}\newline
	\textbf{по дисциплине "Вычислительные алгоритмы"}\newline\newline\newline
	\end{minipage}
\end{center}

\noindent\textbf{Тема} $\underline{\text{Среднеквдратичное приближение.}}$\newline\newline
\noindent\textbf{Студент} $\underline{\text{Романов А.В.}}$\newline\newline
\noindent\textbf{Группа} $\underline{\text{ИУ7-43Б}}$\newline\newline
\noindent\textbf{Оценка (баллы)} $\underline{\text{~~~~~~~~~~~~~~~~~~~~~~~~~~~}}$\newline\newline
\noindent\textbf{Преподаватель} $\underline{\text{Градов В.М.}}$\newline

\begin{center}
	\vfill
	Москва~---~\the\year
~г.
\end{center}
\clearpage
\section{Тема работы}

\textbf{}Построение и программная реализация алгоритма наилучшего среднеквадратичного приближения.
\section{Цель работы}

\textbf{}Получение навыков построения алгоритма метода наименьших квадратов с использованием полинома заданной степени при аппроксимации табличных функций с весами.
\section{Входные данные}

\noindent\textbf{1.} Таблица функции с веми $p_{i}$ с количеством узлов $N$.
\begin{center}
	\begin{tabular}{ |c|c|c| } 
		\hline
		$x$ & $y$ & $\rho$ \\
		\hline
		\multirow{1}{3em} {$x_{i}$} & $   {y_{i}}$  & $\rho_{i}$  \\ 
		\hline
	\end{tabular}
\end{center}


\noindent\textbf{2.} Степень аппроксимирующего полинома -- $n$.
\section{Выходные данные}
\noindent\textbf{}График, на котором изоброжённ аппроксимирующий полином, и точки из исходной таблицы значений. 

\clearpage
\section{Описание алгоритма}
\begin{flushleft}
Под близостью в среднем исходной и аппроксимирующей  функций будем понимать результат оценки суммы 

\begin{equation}
	I = \sum_{i=1}^{N} \rho_{i}[y(x_{i}) - \varphi(x_{i})]^2
	\label{eq:ref}
\end{equation}

$y(x)$ - исходная функция\newline
$\varphi(x)$ - множество функций , принадлежащих линейному пространству функций\newline
$\rho_{i}$ - вес точки\newline

Нужно найти наилучшее приближение, т.е
\begin{equation}
	\sum_{i=1}^{N} \rho_{i}[y(x_{i}) - \varphi(x_{i})]^2 = min
	\label{eq:ref}
\end{equation}

Разложим функцию $\varphi(x)$ по системе линейно независимых функций $\varphi_{k}(x)$:

\begin{equation}
	\varphi(x) = \sum_{k=0}^{N}a_{k}\varphi_{k}(x)
	\label{eq:ref}
\end{equation}

Подставляя (3) в условие (2) получим:
\begin{equation}
	((y - \varphi), (y - \varphi)) = (y, y) - 2\sum_{k=0}^{n}a_{k}(y,\varphi_{k}) + \sum_{k=0}^{n}\sum_{m=0}^{n}a_{k}a_{m}(\varphi_{k},\varphi_{m}) = min
	\label{eq:ref}
\end{equation}

Дифференцируя по $a_{k}$ получаем:
\begin{equation}
	\sum_{i=0}^{n}(x^k, x^m)a_{m} = (y, x^k)
	\label{eq:ref}
\end{equation}
где
$$(x^k, x^m) = \sum_{i=1}^{N}\rho_{i}x_{i}^{k+m}$$
$$(y, x^k) = \sum_{i=1}^{N}\rho_{i}y_{i}x_{i}^{k}$$\newline

Итоговый алгоритм: 

\noindent\textbf{1.} Выбирается степень полинома $n < N$. \newline
\noindent\textbf{2.} Составляется система линейных алгебраических уравнений типа.\newline
\noindent\textbf{3.} В результате решения СЛАУ находятся коэффицинты полинома.
\end{flushleft}

\clearpage % разрыв страницы

\section{Результаты работы программы}
\noindent\textbf{1.} Веса точек равны.

\begin{center}
Исходная таблица: 

\begin{center}
	\begin{tabular}{ | l | l | l | p{1cm} |}
		\hline
		$x_{i}$ & $y_{i}$ & $\rho_{i}$ \\ \hline
		1 & 1 & 1 \\ \hline
		2 & 4 & 1 \\ \hline
		3 & 9 & 1 \\ \hline
		4 & 16 & 1 \\ \hline
		5 & 25 & 1 \\ \hline
		6 & 36 & 1 \\ \hline
		7 & 49 & 1 \\ \hline
		8 & 64 & 1 \\ \hline
		9 & 81 & 1 \\ \hline					
		10 & 100 & 1 \\
		\hline
	\end{tabular}
\end{center}

\includegraphics[scale=0.5]{../../src/n1w1.png} \newline
\includegraphics[scale=0.5]{../../src/n2w1.png} \newline
\includegraphics[scale=0.5]{../../src/n6w1.png} \newline
\end{center}

\clearpage
\noindent\textbf{2.} Веса точек разные.

\begin{center}
	Исходная таблица: 

	\begin{center}
		\begin{tabular}{ | l | l | l | p{1cm} |}
			\hline
			$x_{i}$ & $y_{i}$ & $\rho_{i}$ \\ \hline
			1 & 1 & 0.1 \\ \hline
			2 & 4 & 0.5 \\ \hline
			3 & 9 & 1 \\ \hline
			4 & 16 & 2 \\ \hline
			5 & 25 & 5 \\ \hline
			6 & 36 & 1.9 \\ \hline
			7 & 49 & 3 \\ \hline
			8 & 64 & 0.3 \\ \hline
			9 & 81 & 5 \\ \hline					
			10 & 100 & 0.1 \\
			\hline
		\end{tabular}
	\end{center}

	\includegraphics[scale=0.5]{../../src/weight1.png} \newline\newline
\end{center}

\begin{center}
Синяя прямая - веса точек равны единице.\newline
Красная прямая - веса точек разные.\newline
\end{center}

\clearpage
\section{Ответы на вопросы для защиты ЛР}
\noindent\textbf{1.} Что произойдет при задании  степени полинома $n = N-1$\newline

\noindent\textbf{}Полином будет построен по всем точкам, независимо от того, какие будут веса у точек.\newline

\noindent\textbf{2.} Будет ли работать Ваша программа при $ n >= N$? Что именно в алгоритме требует отдельного анализа данного случая и может привести к аварийной остановке?\newline

\noindent\textbf{}Программа работать будет. Но, по $N$ точкам нельзя построить полином степени $n$, так как в данном случае определитель будет равен нулю. Программа будет работать из-за погрешностей, будут операции с действительными числами.\newline

\noindent\textbf{3.} Получить формулу для коэффициента  полинома $a_{0}$  при степени полинома $n = 0$. Какой смысл имеет  величина, которую представляет данный коэффициент?\newline

\noindent\textbf{}Формула:
$$ \frac{\sum_{i=1}^{N}y_{i}\rho_{i}}{\sum_{i=1}^{N} \rho_{i}} $$\newline

\noindent\textbf{}Значение: математичское ожидание\newline

\noindent\textbf{4.} Записать и вычислить определитель матрицы СЛАУ для нахождения коэффициентов полинома для случая, когда  $n=N=2$. \newline

\noindent\textbf{}Пусть есть таблица:

\begin{center}
	\begin{tabular}{ | l | l | l | p{1cm} |}
		\hline
		$x_{i}$ & $y_{i}$ & $\rho_{i}$ \\ \hline
		$x_{0}$ & $y_{0}$ & $\rho_{1}$ \\ \hline
		$x_{1} $& $y_{1}$ & $\rho_{2}$ \\
		\hline
	\end{tabular}
\end{center}

\noindent\textbf{}Тогда имеем СЛАУ вида:

\begin{equation*}
	\begin{cases}
		(\rho_{0}+\rho_{1})a_{0} + (\rho_{0}x_{0} + \rho_{1}x_{1})a_{1} + (\rho_{0}x_{0}^2 + \rho_{1}x_{1}^2)a_{2} = \rho_{0}y_{0} + \rho_{1}y_{1}\\
		
		(\rho_{0}x_{0} + \rho_{1}x_{1})a_{0} + (\rho_{0}x_{0}^2 + \rho_{1}x_{1}^2)a_{1} + (\rho_{0}x_{0}^3 + \rho_{1}x_{1}^3)a_{2} = \rho_{0}y_{0}x_{0} + \rho_{1}y_{1}x_{1}\\
		
		(\rho_{0}x_{0}^2 + \rho_{1}x_{1}^2)a_{0} + (\rho_{0}x_{0}^3 + \rho_{1}x_{1}^3)a_{1} + (\rho_{0}x_{0}^4 + \rho_{1}x_{1}^4)a_{2} = \rho_{0}y_{0}x_{0}^2 + \rho_{1}y_{1}x_{0}^2\\
	\end{cases}
\end{equation*}

$$ \Delta = (\rho_{0}+\rho_{1})(\rho_{0}x_{0}^2 + \rho_{1}x_{1}^2)(\rho_{0}x_{0}^4 + \rho_{1}x_{1}^4) + (\rho_{0}x_{0} + \rho_{1}x_{1})(\rho_{0}x_{0}^3 + 
 \rho_{1}x_{1}^3)(\rho_{0}x_{0}^2 + \rho_{1}x_{1}^2) + $$
 $$(\rho_{0}x_{0}^2 + \rho_{1}x_{1}^2)(\rho_{0}x_{0} + \rho_{1}x_{1})(\rho_{0}x_{0}^3 + \rho_{1}x_{1}^3) - (\rho_{0}x_{0}^2 + \rho_{1}x_{1}^2)(\rho_{0}x_{0}^2 + \rho_{1}x_{1}^2)(\rho_{0}x_{0}^2 + \rho_{1}x_{1}^2) - $$
  $$(\rho_{0}+\rho_{1})(\rho_{0}x_{0}^3 + \rho_{1}x_{1}^3)(\rho_{0}x_{0}^3 + \rho_{1}x_{1}^3) - (\rho_{0}x_{0} + \rho_{1}x_{1})(\rho_{0}x_{0} + \rho_{1}x_{1})(\rho_{0}x_{0}^4 + \rho_{1}x_{1}^4) = 0$$\newline
\noindent\textbf{}Так как $\Delta = 0$, система решений не имеет, как я и говорил в ответе на вопрос 2.\newline

\clearpage
\section{Код программы}
\noindent\textbf{Файл Main.hs:}
\lstinputlisting[language=Haskell]{../../src/Main.hs}

\noindent\textbf{\newlineФайл Gauss.hs:}
\lstinputlisting[language=Haskell]{../../src/Gauss.hs}

\noindent\textbf{\newlineФайл Approximation.hs:}
\lstinputlisting[language=Haskell]{../../src/Approximation.hs}

\noindent\textbf{\newlineФайл Plot.hs:}
\lstinputlisting[language=Haskell]{../../src/Plot.hs}

\end{document}